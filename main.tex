
\documentclass[
	article,
	12pt,
	oneside,
	a4paper,
	oneside,
	english,
	doc
	%, draftfirst #uncomment if you want a draft document
	%,mask # uncomment if you want a blind review
	]{apa6}
	
\usepackage[
backend=biber,
style=apa6,
sorting=nyt
]{biblatex}


\usepackage{tikz}
\usetikzlibrary{trees}
\usepackage{pgfgantt}
\usepackage{graphicx}
\usepackage{xcolor}
\usepackage{import}
 \usepackage{array}
\usepackage[english]{babel}
%\usepackage[T1]{fontenc}
\usepackage{indentfirst}
\usepackage{color}
\usepackage{booktabs}
\usepackage{graphicx}
%\usepackage[alf]{abntex2cite}

\usepackage{hyperref}
%\usepackage[portuguese]{babel}
\usepackage[utf8]{inputenc}
\usepackage{csquotes}
\usepackage{import}
%\pagestyle{headings}

%Max 10 pages, not including the bibliography
%s: (i) Title,Background and Relevance; (ii) Objectives; (iii) Methodology; (iv) Timetable; (v) U.S.host institution (indicate and justify the U.S. host institution you will attend – must be the same from the Invitation letter) and Bibliography
\addbibresource{Literatura.bib}


\title{NO TITLE YET}
\shorttile{NO TITLE YET}

\author{Matheus Cunha \\ Mariana Meneses\\ Marcus Torres}

\affiliation{Universidade Federal de Pernambuco, Brazil}
\leftheader{Cunha, Menezes e Torres}
%\note{Last version: \today }


\begin{document}
\maketitle

\section{TITLE OPTIONS - WRITE HERE!}

\begin{center}

"Sack the Rich? Exposure to Inequality and Redistributive Attitudes."

"All for Redistribution, as Long as I Don't Pay. Evidence from a Survey Experiment in Brazil." 

"Yay for Redistribution, Nay for More Taxes! Exposure to Inequality and Redistributive Attitudes in Brazil."

"Is Redistribution Worth More Than a Snack? Evidence from a Survey Experiment in Brazil."

\end{center}


%Citation types in apa6

\section{Introduction}

AFTER EVERYTHING ELSE. 

Research question: does exposure to income inequality affect redistributive attitudes? Is this effect conditional on the type of inequality information? 

\section{Relevance}

Inequality likely stands among the gravest problems of the 21st century (CITE).  Inequality degrades generalized trust and even trust in political institutions and politicians (CITE). As inequality grows, money tends to play an increasingly disproportional effect on political competition and policy decisions, disconnecting representatives and their staffs from voters (CITE). Income inequality also strongly correlates to wealth and territorial inequality, unleashing processes of segregation (CITE). Individuals at the lower end of the income distribution become susceptible to myriad social problems. For instance, scholars now point to how growing inequality in the United States is related to the existence of broken communities, deaths of despair and the opioid crisis (CITE).

Several influential theories on Political Science and Economic Politics hold that major institutional and social change depend upon the existence of a redistributive conflict between governing elites and the general population (CITE HERE). Individuals are assumed to hold pro-redistribution attitudes in accordance to their distance from their society's mean income. As the distribution of income is known to be skewed, the majority of any population is consistently below average income. Redistributive conflicts, derived from the average distance in the population, bring about several outcomes of interest, such as public policy responses (CITE) and processes of democratization (CITE). Our work centers on the (often assumed) connection between exposure to inequality and the formation of redistributive preferences. 

\section{Theory and Argument}



\section{Design}


\printbibliography
%\bibliographystyle{apsr} % Alterar para ABNT 
\end{document}